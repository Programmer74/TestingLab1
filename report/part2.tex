\subsection{Постановка задачи}
Провести модульное тестирование указанного алгоритма. Для этого выбрать характерные точки внутри алгоритма, и для предложенных самостоятельно наборов исходных данных записать последовательность попадания в характерные точки. Сравнить последовательность попадания с эталонной.
\subsection{Алгоритм}
Программный модуль для сортировки массива по алгоритму быстрой сортировки \url{http://www.cs.usfca.edu/~galles/visualization/ComparisonSort.html}
\subsection{Исходный код}
\subsubsection{Сортировка}
\lstinputlisting[language=java]{../src/main/java/testing/lab1/QuickSort.java}
\subsubsection{QsortSwapAction.java}
\begin{lstlisting}[language=java]
package testing.lab1;

public class QsortSwapAction {
    private int pivot;
    private int pivotIndex;

    private int leftValue;
    private int leftIndex;

    private int rightValue;
    private int rightIndex;

    public QsortSwapAction(int pivot, int pivotIndex, int leftValue, int leftIndex, int rightValue, int rightIndex) {
        this.pivot = pivot;
        this.pivotIndex = pivotIndex;
        this.leftValue = leftValue;
        this.leftIndex = leftIndex;
        this.rightValue = rightValue;
        this.rightIndex = rightIndex;
    }

    @Override
    public String toString() {
        return "QsortSwapAction{" +
                "pivot=" + pivot +
                ", pivotIndex=" + pivotIndex +
                ", leftValue=" + leftValue +
                ", leftIndex=" + leftIndex +
                ", rightValue=" + rightValue +
                ", rightIndex=" + rightIndex +
                '}';
    }

    @Override
    public boolean equals(Object o) {
        if (this == o) return true;
        if (o == null || getClass() != o.getClass()) return false;

        QsortSwapAction that = (QsortSwapAction) o;

        if (pivot != that.pivot) return false;
        if (pivotIndex != that.pivotIndex) return false;
        if (leftValue != that.leftValue) return false;
        if (leftIndex != that.leftIndex) return false;
        if (rightValue != that.rightValue) return false;
        return rightIndex == that.rightIndex;
    }
    // getters and setters
}
\end{lstlisting}
\subsubsection{QsortSwapActionHistory.java}
\lstinputlisting[language=java]{../src/main/java/testing/lab1/QsortSwapActionHistory.java}
\subsection{Тесты}
\lstinputlisting[language=java]{../src/test/java/testing/lab1/QuickSortTest.java}
\subsection{Обоснование тестового покрытия}
В сортировку передан объект, позволяющий сохранять историю обменов элементов в массиве.
Первый тест просто сравнивает результаты сортировки с упорядоченным набором.
Следующий проверяет, что на большом массиве алгоритм делает один из необходимых шагов.

Дальнейшие тесты направлены на проверку корректности поведения алгоритма на всех стадиях работы.
Группа тестов \texttt{noActionsTest} проверяет, что для отсортированных массивов
(и, в частности, пустого) не производится никаких действий. Тест \texttt{halfSortedTest}
позволяет отследить, как алгоритм ведёт себя на частично отсортированном массиве.
\texttt{reversedTest} демонстрирует работу на отсортированном в обратном порядке
массиве, который является ещё одним вырожденным случаем для алгоритма, подтверждающим
его корректность. \texttt{npeTest} служит для проверки на уведомление об ошибке
(выбросе исключения) в случае передачи алгоритму значения \texttt{null} вместо
ссылки на массив.

Таким образом удалось покрыть алгоритм адекватным набором тестов, демонстрирующих
его корректную работу как на произвольных данных, так и в вырожденных случаях.

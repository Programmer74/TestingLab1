\subsection{Постановка задачи}
Провести модульное тестирование указанного алгоритма. Для этого выбрать характерные точки внутри алгоритма, и для предложенных самостоятельно наборов исходных данных записать последовательность попадания в характерные точки. Сравнить последовательность попадания с эталонной.
\subsection{Алгоритм}
Программный модуль для сортировки массива по алгоритму быстрой сортировки \url{http://www.cs.usfca.edu/~galles/visualization/ComparisonSort.html}
\subsection{Исходный код}
\subsubsection{Сортировка}
\lstinputlisting[language=java]{../src/main/java/testing/lab1/QuickSort.java}
\subsubsection{QsortSwapAction.java}
\lstinputlisting[language=java]{../src/main/java/testing/lab1/QsortSwapAction.java}
\subsubsection{QsortSwapActionHistory.java}
\lstinputlisting[language=java]{../src/main/java/testing/lab1/QsortSwapActionHistory.java}
\subsection{Тесты}
\lstinputlisting[language=java]{../src/test/java/testing/lab1/QuickSortTest.java}
\subsection{Почему так}
А черт его знает.